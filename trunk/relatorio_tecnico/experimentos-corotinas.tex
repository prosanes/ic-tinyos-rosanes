Com o objetivo de comparar o desempenho da implementação de co-rotinas com a biblioteca \textit{TOSThread}, foram
desenvolvidas duas aplicações para implementar o problema do produtor-consumidor.
Uma utilizando \textit{threads} (anexo~\ref{a:appTesteThread}), e outra utilizando co-rotinas (anexo~\ref{a:appTesteCoro}).
Foram utilizados uma linha de execução para o produtor, e outra para o consumidor, e um \textit{buffer} de tamanho
único. Para simular o tempo de processamento da produção e do consumo de uma unidade, foi implementado um laço de cem iterações, onde
cada passo executa uma operação aritmética. Após consumir mil produtos, uma nova linha de execução é ativada, para
calcular o tempo de execução.

Para variar a carga, foram utilizadas diferentes operações aritiméticas.
O tempo de execução foi medido em uma plataforma \textit{MicaZ}, utilizando o temporizador
\textit{Counter<TMicro,uint32\_t>}, utilizando uma precisão de microsegundos.
Os valores medidos não variaram mais de uma unidade entre diferentes execuções.
\begin{center}
    \begin{tabular}{ | l | l | l | l | l | p{5cm} |}
    \hline
    Operação    & x += 1 & x *= 2 & x *= 3 & x *= 5 \\ \hline
    Threads     & 380000 & 490000 & 530000 & 563000 \\ \hline 
    Co-rotinas  & 151000 & 252000 & 289000 & 314000 \\ \hline 
    \end{tabular}
\end{center}
