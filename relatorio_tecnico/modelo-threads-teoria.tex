%!!!explicar esse trecho melhor depois!!!
%Os conceitos já utilizados/implementados por threads que ajudarão na criação de uma interface do corotinas são: 
%Chamadas bloqueantes ao sistema, re-aproveitamento dos serviçõs oferecidos, bloco de controle de threads (TCB), e
%a da troca de contexto.

\textit{TOSThreads} permite programação com threads no TinyOS sem violar ou limitar o modelo de concorrência do
sistema. O TinyOS executa em uma única thread --- no de kernel --- enquanto a aplicação executa 
em uma ou mais threads --- nível de usuário.
Em termos de escalonamento, o kernel tem prioridade máxima, ou seja, a aplicação só executa quando o núcleo do sistema
está ocioso. Ele é responsável pelo escalonamento de tarefas e execução das chamadas de sistemas. 

%!!!explicar esse trecho melhor depois!!!
%A interface entre as threads de usuário e de kernel é feito através de chamadas de sistema bloqueantes. Essas
%chamadas são implementadas aproveitando os serviços já disponíveis do TinyOS. 
%São responsáveis por manter o estado do serviço 
%\textit{split-phase} que será usado, bloquear a thread que a invocou e acordar a thread do kernel.

Três tipos de contextos de execução passam a existir: tarefas, interrupções e threads. Tarefas e interrupções podem
interromper threads de aplicação, mas não o contrário. Threads tem preempção entre elas, 
de modo que é necessário o uso de primitivas de sincronização. 
As opções fornecidas são \textit{mutex}, semáforos, barreiras, variáveis de condição, e contador
bloqueante. Esta última foi desenvolvida especialmente para o TinyOS. Seu uso se dá de forma que a thread fica bloqueada
até o contador atingir um número arbitrário, enquanto outras threads podem incrementar ou decrementar esta variável
através de uma interface específica.
%!!!explicar esse trecho melhor depois!!!
%O TinyOS retoma o controle sobre a aplicação de dois modos diferentes. No primeiro, uma aplicação faz uma chamada de
%sistema que posta uma tarefa para processar o serviço. No segundo modo, um manipulador de interrupção posta uma tarefa.
%Porém, neste caso o TinyOS só acorda depois de terminada a execução da interrupção.
O escalonador de threads utiliza uma política \textit{Round-Robin} com um tempo de 5 milisegundos. É ele que oferece
toda a interface para manipulação de threads, como pausar, criar e destruir. 
%!!!explicar esse trecho melhor depois!!!
%É interessante notar que o escalonador não
%existe em um contexto de execução específico, seu contexto depende de quem utilizou sua interface.

As threads podem ser estáticas ou dinâmicas. A diferença está no momento de criação da pilha e do bloco de controle da
thread. Nas threads estáticas a criação é feita em tempo de compilação, enquanto nas threads dinâmicas 
a criação é feita em tempo de execução. O bloco de controle, também chamado de
\textit{Thread Control Block} (TCB), contém informações essenciais da thread, como seu identificador, seu estado de
execução, o valor dos registradores (para troca de contexto), entre outras\cite{TEP134}.
A troca de contexto é feita por códigos específicos para cada plataforma. 


