\documentclass[a4paper,onecolumn, 10pt]{article}
\usepackage[utf8]{inputenc}
\usepackage[portuguese,brazil]{babel}
\usepackage{ae}

\usepackage[pdftex]{graphicx} % Exporta para pdf e aceita figuras
%\usepackage{indentfirst} % Identa primeiro paragrafo
\usepackage{textcomp}
\usepackage[margin=30mm]{geometry}
\usepackage[pdfauthor={Pedro Rosanes},% Insere metadados com o nome do autor
	    pdftitle={Relatório dos Componentes de Escalonamento},% Título que será mostrado na janela do PDF
	    pdftex]{hyperref} % Usa hiperlinks no decorrer do texto
\usepackage{color}
\usepackage{listings} 
%\lstset{language=C++}
\lstset{ %
    language=C++,                % choose the language of the code
        basicstyle=\footnotesize,       % the size of the fonts that are used for the code
        backgroundcolor=\color{white},  % choose the background color. You must add
        showspaces=false,               % show spaces adding particular underscores
        showstringspaces=false,         % underline spaces within strings
        showtabs=false,                 % show tabs within strings adding particular underscores
        frame=single,                   % adds a frame around the code
        tabsize=2,                  % sets default tabsize to 2 spaces
        captionpos=b,                   % sets the caption-position to bottom
        breaklines=true,                % sets automatic line breaking
        breakatwhitespace=false,        % sets if automatic breaks should only happen at whitespace
        title=\lstname,                 % show the filename of files included with \lstinputlisting;
    % also try caption instead of title
        escapeinside={\%*}{*)},         % if you want to add a comment within your code
        morekeywords={*,...}            % if you want to add more keywords to the set
}

%\parskip 7.2pt           % sets spacing between paragraphs
\renewcommand{\baselinestretch}{1.5}

\title{Relatório dos Componentes de Escalonamento}
\author{Pedro Rosanes}
%\date{}

\begin{document}

\maketitle


%\begin{titlepage}
%\maketitle
%\tableofcontents % Sumário
%\end{titlepage}


%%%%%%%%%%%%%%%%%%%%%%%%%%%%%%%%%%%%%%%%%%%%%%%%%%%%%%%%%%%%%%%%%%%%%%%%%%%%%%%%%%%%%%%%%%%%%%%%%%%%%%%%%%%%%%%%%%%%%%%%%%%%%%%%%%%%%%%%%%%%%%%%%%%%%%%%%%%%%%%%%%%%%%%%%%%%%%%%%%%%%%%%%%%%%%%%%%%%%%%%%%%%%%%%%%%%%%%%%%%%%%%%

\section{Análise do Escalonador Padrão}\label{escalonadorpadrao}

O escalonador padrão adota uma política FIFO. Ele provê as interfaces \textit{Scheduler} e \textit{TaskBasic}.
As tarefas se conectam ao escalonador através da \textit{TaskBasic}. Ao compilar um programa em NesC, todas tarefas
básicas viram uma interface desse tipo. Porém, para se diferenciarem é criado um parâmetro na interface
\footnote{Para mais informações sobre interfaces parametrizadas olhar o livro TinyOS Programming\cite[s. 8.3 e 9]{tinyosprogramming}.}.

A fila implementada \textbf{não} funciona como um vetor circular. Existe um 'ponteiro' para a cabeça e um para o rabo,
além de um vetor de tamanho 255 (O que limita a quantidade máxima de tarefas). A cabeça contém o identificador do 
primeiro na fila. A célula cujo index corresponde ao \textit{id} do primeiro contém o \textit{id} do segundo, e assim em 
diante. Chega-se ao fim da fila quando a célula contém um identificador de vázio.

%\includegraphics[scale=0.5]{fila.png}

\section{Análise do Escalonador \textit{Earliest Deadline First}}\label{escalonadoredf}

Este escalonador aceita tarefas com deadline e elege as que menor \textit{deadline} para executar. A interface usada para criar
esse tipo de tarefas é \textit{TaskDeadline}. O \textit{deadline} é passado por parâmetro pela função \textit{postTask}.
As \textit{TaskBasic} também são aceitas como recomendado pelo TEP
106\cite{tep106}.

Em contraste, o escalonador não segue outra recomendação. Ele não elimina a possibilidade de \textit{starvation} pois as tarefas
básicas só são atendidas quando não há nenhuma com \textit{deadline} para ser atendida. A fila de prioridades é
implementada da mesma forma que a do escalonador padrão\ref{escalonadorpadrao}, a única mudança está na inserção. Para
inserir, a fila é percorrida do começo até o fim, procurando-se o local exato de inserção. O problema é que essa
operação custa $\bigcirc(n)$. Isso poderia ser evitado implementando uma \textit{heap} que tem complexidade de inserção
de $\bigcirc(\log n)$.

A princípio tive problemas com o componente \textit{Counter32khzC}, 
pois ele não existe para o \textit{micaz}. Para poder compilar o
escalonador tive de tirá-lo. Ele era usado para calcular a hora atual, e somar ao deadline. Sem esse componente, temos
um escalonador de prioridades (mínimo). 

Descobri que tarefas se comportam de forma diferente no simulador. Portanto tive de adicionar funções que lidam com
eventos no \textit{tossim}. Essas funções foram retiradas do arquivo
\textit{opt/tinyos-2.1.1/tos/lib/tossim/SimSchedulerBasicP.nc}.
Primeiro é preciso adicionar ao \textit{Scheduler}:
\begin{lstlisting}[frame=single]
  bool sim_scheduler_event_pending = FALSE;
  sim_event_t sim_scheduler_event;

  int sim_config_task_latency() {return 100;}
  
  void sim_scheduler_submit_event() {
    if (sim_scheduler_event_pending == FALSE) {
      sim_scheduler_event.time = sim_time() + sim_config_task_latency();
      sim_queue_insert(&sim_scheduler_event);
      sim_scheduler_event_pending = TRUE;
    }
  }

  void sim_scheduler_event_handle(sim_event_t* e) {
    sim_scheduler_event_pending = FALSE;
    if (call Scheduler.runNextTask()) {
      sim_scheduler_submit_event();
    }
  }

  void sim_scheduler_event_init(sim_event_t* e) {
    e->mote = sim_node();
    e->force = 0;
    e->data = NULL;
    e->handle = sim_scheduler_event_handle;
    e->cleanup = sim_queue_cleanup_none;
  }
\end{lstlisting}
Depois, no \textit{Scheduler.init()} adicione:
\begin{lstlisting}[frame=single]
  sim_scheduler_event_pending = FALSE;
  sim_scheduler_event_init(&sim_scheduler_event);
\end{lstlisting}
E por ultimo, no \textit{Scheduler.postTask()}, caso a tarefa tenha sido colocada na fila, adicione:
\begin{lstlisting}[frame=single]
  sim_scheduler_submit_event();
\end{lstlisting}


\pagebreak

\begin{thebibliography}{9}
\bibitem{tinyosprogramming} P. Levis e D. Gay. TinyOS Programming, capítulo 11, 2009.
\bibitem{tep106} P. Levis e C. Sharp. TEP 106: Schedulers and Tasks.
                    \url{http://www.tinyos.net/tinyos-2.x/doc/html/tep106.html}
\bibitem{bootsequence} Boot Sequence, TinyOS Tutorial. \url{http://docs.tinyos.net/index.php/Boot_Sequence}
\end{thebibliography}

\end{document}
