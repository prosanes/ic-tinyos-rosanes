O modelo que decidimos implementas foi um descrito por Ana Moura em sua tese de doutorado\cite[s. 6.2]{Moura/04}.
Neste modelo existe uma co-rotina principal que é responsável por escalonar as outras co-rotinas.

Nossa implementação utilizou como base a extensão \textit{TOSThreads}, vista na seção \ref{TOSThreads}. O primeiro passo
foi criar uma cópia do diretório desta extensão e um novo \textit{target} referente a este diretório para o \textit{make} do TinyOS.

Na implementação do \textit{TOSThreads} existem dois casos em que ocorre preempção: termino do \textit{timeslice} e
acontecimento de uma interrupção de hardware. Portanto foi preciso modificar esses dois casos.
A primeira alteração foi retirar o limite de tempo de execução de cada thread. Para isso o temporizador responsável por
essa contagem foi desabilitado.
A segunda alteração foi criar um novo tipo de interrupção, que chamamos de interrupção curta. Originalmente, no
\textit{TOSThreads}, quando o tratador de interrupção postava uma tarefa, o kernel assumia o controle, executava a
tarefa e escalonava a próxima thread da fila. Na nossa implementação, após o kernel executar a tarefa, a thread que foi
originalmente interrompida volta a executar. Para isso, foi criado um novo comando no escalonador de threads:
\textit{brieflyInterruptCurrentThread()}
\begin{lstlisting}
async command error_t ThreadScheduler.brieflyInterruptCurrentThread() {
    atomic {
        if(current_thread->state == TOSTHREAD_STATE_ACTIVE) {
            briefly_interrupted_thread = current_thread;
            briefly_interrupted_thread->state =
                TOSTHREAD_STATE_BRIEFLYINTERRUPTED;
            interrupt(current_thread);
            return SUCCESS;
        }
        return FAIL;
    }
}

/* schedule_next_thread()
 * This routine does the job of deciding which thread should run next.
 * Should be complete as is.  Add functionality to getNextThreadId() 
 * if you need to change the actual scheduling policy.
 */
    void scheduleNextThread() {
        if(tos_thread->state == TOSTHREAD_STATE_READY)
            current_thread = tos_thread;
        else if (briefly_interrupted_thread != NULL)
        {
            current_thread = briefly_interrupted_thread;
            briefly_interrupted_thread = NULL;
        }
        else
            current_thread = call ThreadQueue.dequeue(&ready_queue);

        current_thread->state = TOSTHREAD_STATE_ACTIVE;
    }

\end{lstlisting}

Uma vez excluída a preempção, o proxímo passo foi modificar a interface da thread para permitir passagem de controle ao
escalonador. Para isso, foi criado o comando \textit{yield()}. É importante notar que este comando pode ser chamado em 
qualquer ponto do programa, porém só deve ser chamado dentro da thread ao qual o comando se refere.
\begin{lstlisting}
//Arquivo: interfaces/Thread.nc
interface Thread {
    ...
    command error_t yield();
    ...
}

//Arquivo: system/StaticThreadP.nc
module StaticThreadP {
    ...
    command error_t Thread.yield[uint8_t id]() { 
        return call ThreadScheduler.interruptCurrentThread(); 
    } 
    ...
}
\end{lstlisting}

%!!! Que tal adicionar um apendice mostrando o passo-a-passo da implementação de co-rotinas? !!!


