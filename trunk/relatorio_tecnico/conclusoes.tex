As redes de sensores sem fio podem ser aplicadas em diversas áreas, por exemplo, monitoramento de oscilações 
e movimentos de pontes, observação de vulcões ativos, previsão de incêndio em florestas, entre outras. 
Muitas dessas aplicações podem atingir alta complexidade, exigindo a construção de algoritmos robustos, 
como roteamento de pacotes diferenciado.
Os escalonadores desenvolvidos neste trabalho poderão ajudar os desenvolvedores dessas aplicações complexas,
oferecendo maior flexibilidade no projeto das soluções, como, por exemplo, 
a possibilidade de priorizar certas atividades da aplicação (comunicação via rádio ou serial, sensoriamento, etc.).
Porém é preciso analisar se o ganho em flexibilidade, oferecido pelo escalonador, irá compensar o
\textit{overhead} gerado.
Pretendemos realizar ainda outros experimentos com os escalonadores desenvolvidos, considerando diferentes
tipos de aplicações. 

Sem um fluxo contínuo de execução, sobre a perspectiva do programador, as aplicações grandes ficam difíceis de
implementar e entender. O modelo de \textit{threads} oferecido no 
TinyOS 2.1.X\cite{TEP134} facilita este problema. Entretanto, por ser um modelo preemptivo, o custo de
gerência das threads pode implicar em queda de desempenho das aplicações. 
Com a implementação de um mecanismo de cooperação baseado em co-rotinas pretendemos
oferecer uma alternativa a mais para o programador.


