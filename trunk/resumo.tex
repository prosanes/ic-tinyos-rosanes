\documentclass[a4paper,onecolumn, 10pt]{article}
\usepackage[utf8]{inputenc}
\usepackage[portuguese,brazil]{babel}
\usepackage{ae}

\usepackage[pdftex]{graphicx} % Exposta para pdf e aceita figuras
%\usepackage{indentfirst} % Identa primeiro paragrafo
\usepackage{textcomp}
\usepackage[margin=30mm]{geometry}
\usepackage[pdfauthor={Pedro Rosanes},% Insere metadados com o nome do autor
	    pdftitle={Modelo de Concorrência e Mecanismos de Gerência de Tarefas do TinyOS},% Título que será mostrado na janela do PDF
	    pdftex]{hyperref} % Usa hiperlinks no decorrer do texto

%\parskip 7.2pt           % sets spacing between paragraphs
\renewcommand{\baselinestretch}{1.5}

\title{Modelo de Concorrência e Mecanismos de Gerência de Tarefas do TinyOS}
\author{Pedro Rosanes}
%\date{}

\begin{document}

\maketitle


%\begin{titlepage}
%\maketitle
%\tableofcontents % Sumário
%\end{titlepage}


%%%%%%%%%%%%%%%%%%%%%%%%%%%%%%%%%%%%%%%%%%%%%%%%%%%%%%%%%%%%%%%%%%%%%%%%%%%%%%%%%%%%%%%%%%%%%%%%%%%%%%%%%%%%%%%%%%%%%%%%%%%%%%%%%%%%%%%%%%%%%%%%%%%%%%%%%%%%%%%%%%%%%%%%%%%%%%%%%%%%%%%%%%%%%%%%%%%%%%%%%%%%%%%%%%%%%%%%%%%%%%%%

\section{Sequência de Inicialização}\label{sequenciadeinicializacoes}
O principal componente do TinyOS responsável por botar o sistema no ar é o \textit{MainC}. Para isso, ele inicializa o escalonador de tarefas, os componentes de hardware e software. Primeiro é configurado o sistema de memória e escolhido o modo de processamento. Com esses pré-requisitos básicos estabelecidos é inicializado o escalonador, para permitir que as proximas etapas possam postar tarefas.

O segundo passo é inicializar o hardware como um todo, permitindo a operabilidade da plataforma. Alguns exemplos são configuração de pinos de entrada e saida, calibração do clock e dos LEDs. Como esta etapa exige códigos específicos para cada tipo de plataforma, o \textit{MainC} se liga ao componente \textit{PlataformC} que implementa o programa necessário.

O terceiro passo trata o software. Além de configurar os aplicativos básicos do sistema, como o temporizador, também é preciso inicializar os programas do usuário. Portanto, se um componente do usuário precisa ser inicializado basta amarrá-lo ao \textit{SoftwareInit}. Assim o TinyOS se responsabiliza por executar este código.

Por ultimo, quando tudo é concluido, o \texit{MainC} sinaliza que a inicialização concluiu, através do sinal \textit{Boot.booted()}. Isso permite que os componentes rodem. E finalmente, o TinyOS entra no seu laço principal, no qual o escalonador espera por tarefas, e as executa. É importante notar que durante todo este processo, as interrupções do sistema ficam desabilitadas.

%%%%%%%%%%%%%%%%%%%%%%%%%%%%%%%%%%%%%%%%%%%%%%%%%%%%%%%%%%%%%%%%%%%%%%%%%%%%%%%%%%%%%%%%%%%%%%%%%%%%%%%%%%%%%%%%%%%%%%%%%%%%%%%%%%%%%%%%%%%%%%%%%%%%%%%%%%%%%%%%%%%%%%%%%%%%%%%%%%%%%%%%%%%%%%%%%%%%%%%%%%%%%%%%%%%%%%%%%%%%%%%%

\section{Mecânismos de Gerência de Tarefas}\label{mecanismosdegerencia}
\textit{Tasks}, ou tarefas, têm duas propriedades importantes. Elas não são preemptivas entre si, e são executadas de forma adiada. Isso significa que ao postar uma tarefa, o fluxo de execução continua, sem desvio, e ela só será processada depois. As \textit{tasks} básicas não tem parâmetro, apesar de ser possível fazê-las receber parâmetros, criando uma interface e amarrando-a ao componente do escalonador.

O responsável por gerenciar e escalonar tarefas no \textit{TinyOS} é o componente \textit{TinySchedulerC}. O escalonador padrão adota uma política \textit{First-in First-out} para agendar as tarefas. Ele também cuida de parte do gerenciamento de energia, pois bota a CPU em um estado de baixo consumo se não há nada para ser executado.

É possível mudar a política de gerenciamento de tarefas substituindo o escalonador padrão. Para isto, basta adicionar uma configuração com o nome \textit{TinySchedulerC} no diretório da aplicação e amarrá-la ao componente responsável pela implementação. Qualquer novo escalonador tem de aceitar a interface das \textit{tasks} padrões, e garantir a execução de todas as tarefas, sem permitir \textit{Starvation}

%%%%%%%%%%%%%%%%%%%%%%%%%%%%%%%%%%%%%%%%%%%%%%%%%%%%%%%%%%%%%%%%%%%%%%%%%%%%%%%%%%%%%%%%%%%%%%%%%%%%%%%%%%%%%%%%%%%%%%%%%%%%%%%%%%%%%%%%%%%%%%%%%%%%%%%%%%%%%%%%%%%%%%%%%%%%%%%%%%%%%%%%%%%%%%%%%%%%%%%%%%%%%%%%%%%%%%%%%%%%%%%%

\section{Modelo de Concorrência}\label{modelodeconcorrencia}
O \textit{TinyOS} mantém os problemas de concorrência bem simples, qualquer possível condição de corrida é detectada em tempo de compilação. Para que isso seja possível, o código em nesC é dividido em dois tipos:
\begin{description}
  \item[Código Assíncrono]
  Código alcançável a partir de pelo menos um tratador de interrupção.
  \item[Código Síncrono]
  Código alcançavel somente a partir de \textit{tasks}.
\end{description}

Eventos e comandos que podem ser sinalizados ou chamados a partir de um trador de interrupção são códigos assíncronos. Eles podem interromper outros eventos, comandos e \textit{tasks}. Por isso devem ser marcados como \textit{async} no código fonte. O problema aparece quando variáveis compartilhadas são acessadas por esse tipo de código. Para contornar isso, deve-se usar o comando \textit{atomic} ou \textit{Power locks}.

O comando \textit{atomic} permite que um trecho de instruções possa ser executado sem ser interrompido. Dois fatos importantes surgem com o seu uso, primeiro a ativação e desativação de interrupções consome ciclos de CPU. Segundo, longos trechos atômicos podem atrasar outras interrupções, portanto é preciso tomar cuidado ao chamar outros componentes a partir desses blocos.

Algumas vezes é preciso usar um certo hardware por um longo tempo, sem compartilhá-lo. Como a necessidade de atomicidade não está no processador e sim no hardware, pode-se conceder sua exclusividade a somente um 'usuário' através de \textit{Power locks}. Para isso, primeiro é feito um pedido através de um comando, depois quando o recurso desejado estiver disponível, um evento é sinalizado. Assim não há bloqueio de execução, como em semáforos. Existe a possibilidade de requisição imediata. Nesse caso nenhum evento será sinalizado: se o recurso não estiver protegido, ele será imediatamente cedido, caso contrário, o comando retornará falso. \textit{Power Locks} têm três sub-componentes: Um abitrador que gerência as prioridades dos pedidos, um gerênciador de energia e um configurador que ajusta o hardware de acordo com o cliente.

\pagebreak

\begin{thebibliography}{99}

\bibitem{tutorial6} Tutorial 6, Boot Sequence. http://docs.tinyos.net/index.php/Boot_Sequence 

\bibitem{tep106} P. Levis e C. Sharp. TEP 106: Schedulers and Tasks. http://www.tinyos.net/tinyos-2.x/doc/html/tep106.html
\bibitem{tep107} P. Levis. TEP 107: TinyOS 2.x Boot Sequence. http://www.tinyos.net/tinyos-2.x/doc/html/tep107.html
\bibitem{tinyosprogramming} P. Levis e D. Gay. TinyOS Programming, capítulo 11, 2009.
\end{thebibliography}

\end{document}
