As tarefas, por serem procedimentos adiados, necessitam de algoritmos de escalonamento. Estes algoritmos também não
podem ser preemptivos, devido a natureza das tarefas do TinyOS.
O algoritmo mais simples, e também o padrão do TinyOS, é o \textit{First-Come, First-Served}, onde as tarefas são
atendidas segundo a ordem de chegada. Tem um \textit{overhead} mínimo, e não gera \textit{starvation}. Pode ter um tempo de
resposta alto, se houver uma grande quantidade de tarefas na fila.

Escalonamento utilizando \textit{deadline} é muito usado em sistemas operacionais de tempo real. Neste algoritmo
a próxima tarefa a ser executada é aquela com menor prazo (\textit{deadline}). As diversas variações deste algoritmo utilizam
parâmetros como: 
\begin{itemize}
    \item Tempo de entrada na fila de prontos.
    \item Prazo para começar a tarefa.
    \item Prazo para terminar a tarefa.
    \item Tempo de processamento.
    \item Recursos utilizados.
    \item Prioridade.
    \item Existência de preempção.
\end{itemize}
Porém, o principal parâmetro utilizado pelos algoritmos é a existência ou não de preempção. Caso não exista preempção,
faz mais sentido utilizarmos, no escalonamento, o prazo para começar a tarefa. Caso exista preempção, o prazo para
terminar a tarefa é utilizado~\cite{Stallings/04}.
Um \textit{overhead} bem maior passa a existir, porém o tempo de resposta pode ser aproximadamente estipulado pela
própria tarefa.
% !!! Talvez devesse melhorar esta parte !!!

Em um escalonamento de prioridade fixa, cada tarefa indica, no momento de entrada para fila de prontos, sua importância em relação às
outras tarefas. Nestes algoritmos podemos ter preempção por parcela de tempo, na entrada de outras tarefas, ou não ter preempção. 
No primeiro tipo, pode existir um \textit{overhead} desnecessário quando o \textit{time-slice} da tarefa atual terminou,
porém não existe nenhuma outra com prioridade maior. O segundo tipo resolve este problema: Se existe uma ordem de
tarefas na fila, esta ordem só pode mudar caso uma nova tarefa entre.
Quando não há preempção, a troca de tarefa só ocorre no final da que está executando.
Neste escalonamento também há um \textit{overhead} grande, devido a ordenação das tarefas na fila. Também é introduzido
a possibilidade de \textit{starvation} e um alto tempo de resposta para tarefas de baixa prioridade.

O escalonamento de multi-nível é um caso especial do de prioridade fixa. 
Cada tarefa determina seu nível de prioridade em tempo de compilação. Onde cada nível de prioridade tem uma fila, com 
política \textit{First-in First-out}, e as filas mais importantes devem ser atendidas por completo para que outra sejam
atendidas.

O escalonamento de prioridade dinâmica visa eliminar a possibilidade de \textit{starvation}. Neste caso, a tarefa ainda
indica sua importância no momento de entrada para fila de prontos. Porém, as tarefas que estão esperando para executar
aumentam de prioridade toda vez que não são atendidas. Apesar disto aumentar significativamente o \textit{overhead}, o
\textit{starvation} é eliminado.

