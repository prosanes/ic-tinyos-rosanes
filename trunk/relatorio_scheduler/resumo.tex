\documentclass[a4paper,onecolumn, 10pt]{article}
\usepackage[utf8]{inputenc}
\usepackage[portuguese,brazil]{babel}
\usepackage{ae}

\usepackage[pdftex]{graphicx} % Exporta para pdf e aceita figuras
%\usepackage{indentfirst} % Identa primeiro paragrafo
\usepackage{textcomp}
\usepackage[margin=30mm]{geometry}
\usepackage[pdfauthor={Pedro Rosanes},% Insere metadados com o nome do autor
	    pdftitle={Relatório dos Componentes de Escalonamento},% Título que será mostrado na janela do PDF
	    pdftex]{hyperref} % Usa hiperlinks no decorrer do texto

%\parskip 7.2pt           % sets spacing between paragraphs
\renewcommand{\baselinestretch}{1.5}

\title{Relatório dos Componentes de Escalonamento}
\author{Pedro Rosanes}
%\date{}

\begin{document}

\maketitle


%\begin{titlepage}
%\maketitle
%\tableofcontents % Sumário
%\end{titlepage}


%%%%%%%%%%%%%%%%%%%%%%%%%%%%%%%%%%%%%%%%%%%%%%%%%%%%%%%%%%%%%%%%%%%%%%%%%%%%%%%%%%%%%%%%%%%%%%%%%%%%%%%%%%%%%%%%%%%%%%%%%%%%%%%%%%%%%%%%%%%%%%%%%%%%%%%%%%%%%%%%%%%%%%%%%%%%%%%%%%%%%%%%%%%%%%%%%%%%%%%%%%%%%%%%%%%%%%%%%%%%%%%%

\section{Análise do Escalonador Padrão}\label{escalonadorpadrao}

O escalonador padrão adota uma política FIFO. Ele provê as interfaces \textit{Scheduler} e \textit{TaskBasic}. As tarefas se conectam ao escalonador através da \textit{TaskBasic}. Ao compilar um programa em NesC, todas tarefas básicas viram uma interface desse tipo. Porém, para se diferenciarem é criado um parâmetro na interface\footnote{Para mais informações sobre interfaces parametrizadas olhar o livro TinyOS Programming\cite[s. 8.3 e 9]{tinyosprogramming}.}.

A fila implementada \textbf{não} funciona como um vetor circular. Existe um 'ponteiro' para a cabeça e um para o rabo, além de um vetor de tamanho 255 (O que limita a quantidade máxima de tarefas). A cabeça contém o identificador do primeiro na fila. A célula cujo index corresponde ao \textit{id} do primeiro contém o \textit{id} do segundo, e assim em diante. Chega-se ao fim da fila quando a célula contém um identificador de vázio.

%\includegraphics[scale=0.5]{fila.png}

\pagebreak

\begin{thebibliography}{9}
\bibitem{tinyosprogramming} P. Levis e D. Gay. TinyOS Programming, capítulo 11, 2009.
\end{thebibliography}

\end{document}
