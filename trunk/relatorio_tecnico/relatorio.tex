\documentclass[a4paper,onecolumn, 10pt]{article}
\usepackage[utf8]{inputenc}
\usepackage[portuguese,brazil]{babel}
\usepackage{ae}

\usepackage[pdftex]{graphicx} % Exposta para pdf e aceita figuras
%\usepackage{indentfirst} % Identa primeiro paragrafo
\usepackage{textcomp}
\usepackage[margin=30mm]{geometry}
\usepackage[pdfauthor={Pedro Rosanes},% Insere metadados com o nome do autor
	    pdftitle={Abstrações de programação distribuída para redes de sensores sem fio},% Título que será mostrado na janela do PDF
	    pdftex]{hyperref} % Usa hiperlinks no decorrer do texto

%\parskip 7.2pt           % sets spacing between paragraphs
\renewcommand{\baselinestretch}{1.5}

\title{Abstrações de programação distribuída para redes de sensores sem fio}
\author{Bolsista: Pedro Rosanes \and Orientador: Silvana Rossetto \and Departamento de Ciência da Computação}
\date{}

\begin{document}

\maketitle


%\begin{titlepage}
%\maketitle
%\tableofcontents % Sumário
%\end{titlepage}


%%%%%%%%%%%%%%%%%%%%%%%%%%%%%%%%%%%%%%%%%%%%%%%%%%%%%%%%%%%%%%%%%%%%%%%%%%%%%%%%%%%%%%%%%%%%%%%%%%%%%%%%%%%%%%%%%%%%%%%%%%%%%%%%%%%%%%%%%%%%%%%%%%%%%%%%%%%%%%%%%%%%%%%%%%%%%%%%%%%%%%%%%%%%%%%%%%%%%%%%%%%%%%%%%%%%%%%%%%%%%%%%

\section{Resumo}\label{resumo}
Redes de Sensores Sem Fio (RSSFs) caracterizam-se pela formação de aglomerados de pequenos 
dispositivos que, atuando em conjunto, permitem monitorar ambientes físicos ou processos de 
produção com elevado grau de precisão. O desenvolvimento de aplicações que permitam explorar 
o uso dessas redes requer o estudo e a experimentação de protocolos, algoritmos e modelos de 
programação que se adequem às suas características e exigências particulares, entre elas, uso
de recursos limitados, adaptação dinâmica das aplicações, e a necessidade de integração com
outras redes, como a Internet. O objetivo deste projeto é criar abstrações de programação para
o sistema operacional TinyOS (um dos principais sistemas operacionais usados na pesquisa com 
RSSFs), visando facilitar a construção de abstrações de programação de nível mais alto que 
auxiliem o desenvolvimento de aplicações nessa área.

\section{Introdução}\label{intro}
Sistemas projetados para os dispositivos que formam as redes de sensores devem lidar apropriadamente
com as restrições e características particulares desses ambientes. A arquitetura adotada pelo TinyOS 
(Hill/00) prioriza fortemente o tratamento dessas restrições em detrimento da simplicidade oferecida
para o desenvolvimento de aplicações. Entretanto, de acordo com Levis e Culler (Levis/02), para que
as redes de sensores sejam de fato adotadas é preciso que elas sejam mais fáceis de usar. A experiência
de outros pesquisadores (Cheong/03, Han/05, Kasten/05, Dunkels/05), e a nossa experiência particular
desenvolvendo componentes TinyOS (Costa/05), confirmam a importância de propor uma interface de programação
mais adequada.\cite{tesesilvana}

\section{Metodologia / Teoria}\label{metodologia}
A seguir seguem, em ordem cronologica, os estudos e experimentos feitos.

\subsection{Fundamentos das RSSFs com o TinyOS}
Estudo feito sobre o sistema operacional TinyOS, sua linguagem de programação nesC e o simulador TOSSIM.
Como fonte do aprendizado foram usados o minicurso desenvolvido pela professora Silvana Rosseto\cite{minicurso}, o material oferecido
pela página do TinyOS\cite{siteTinyOS}, e o livro \textit{TinyOS Programming}, de Levis e Gay\cite{tinyosprogramming}.

\subsection{Concorrência e Gerência de Tarefas em Sistemas Operacionais}
Antes de estudar o modelo de concorrência e gerência de tarefas específica do TinyOS, foi feita uma revisão para
sistemas operacionais em geral. O matérial utilizado foi o livro do professor Carlos Maziero (PUCPR)\cite{maziero}.

\subsection{Concorrência e Gerência de Tarefas no TinyOS}
********Texto desenvolvido******* \\
********Escalonadores Desenvolvidos******** \\
********Relatorio Técnico******* \\

\subsection{Modelo de Threads do TinyOS}
*******Estudo do modelo de Threads******** \\
*******Texto desenvolvido (Atualizar ASM)******* \\
*******Aplicacoes desenvolvidas****

\subsection{Corotinas}
*******Estudo sobre corotinas**** \\





\pagebreak

\begin{thebibliography}{99}

\bibitem{minicurso} Minicurso sobre TinyOS, Silvana Rosseto,
    \url{http://www.dcc.ufrj.br/~silvana/curso-tinyos-ufes2010/} 
\bibitem{siteTinyOS} TinyOS Documentation Wiki, \url{http://docs.tinyos.net/index.php/Main_Page}
\bibitem{tinyosprogramming} TinyOS Programming, Philip Levis, David Gay.
\bibitem{maziero} Livro de Sistemas Operacionais, Carlos Maziero,
    \url{http://www.ppgia.pucpr.br/~maziero/doku.php/so:livro_de_sistemas_operacionais}

\end{thebibliography}

\end{document}
