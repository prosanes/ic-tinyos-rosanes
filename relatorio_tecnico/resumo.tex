Redes de Sensores Sem Fio (RSSFs) são formadas por pequenos dispositivos de sensoreamento, com 
espaço de memória e capacidade de processamento limitados, fonte de energia esgotável e comunicação sem fio.
O sistema operacional mais usado na programação desses dispositivos é o TinyOS, um sistema leve, 
projetado especialmente para consumir pouca energia, um dos requisitos mais importante para RSSFs. 
%O modelo  programação adotado pelo TinyOS é orientada a componentes (para facilitar a
%reutilização de código) e basedo em eventos (para economizar memória), 
O modelo de programação adotado pelo TinyOS prioriza o atendimento de interrupções.
Em função disso, as operações são normalmente divididas em duas fases: uma para envio
do comando, e outra para o tratamento da resposta (evento sinalizado via interrupção). 
Esse modelo de programação, baseado em eventos, quebra o fluxo de execução normal, dificultando a
tarefa dos desenvolvedores de aplicações. 
Para que os tratadores de eventos (interrupções) sejam curtos, tarefas maiores são
postergadas para execução futura e, para evitar concorrência entre elas,
as tarefas são executadas em sequência, uma após a outra (i.e., uma tarefa só é iniciada após a tarefa
anterior ser concluída).  
O objetivo deste trabalho é propor e implementar políticas alternativas de escalonamento de tarefas
para o TinyOS visando a construção de abstrações de programação de nível mais alto que 
facilitem o desenvolvimento de aplicações nessa área.
